\documentclass{standalone}
\usepackage{luacode}
\usepackage{tikz}

\title{Primes}
\author{Peter Rowlett}
\date{June 2020}

% define some colours
% http://latexcolor.com/
\definecolor{aureolin}{rgb}{0.99, 0.93, 0.0}
\definecolor{amethyst}{rgb}{0.6, 0.4, 0.8}
\definecolor{beaublue}{rgb}{0.74, 0.83, 0.9}
\definecolor{applegreen}{rgb}{0.55, 0.71, 0.0}
\definecolor{bleudefrance}{rgb}{0.19, 0.55, 0.91}
\definecolor{atomictangerine}{rgb}{1.0, 0.6, 0.4}
\definecolor{babypink}{rgb}{0.96, 0.76, 0.76}
\definecolor{blizzardblue}{rgb}{0.67, 0.9, 0.93}
\definecolor{blue-violet}{rgb}{0.54, 0.17, 0.89}
\definecolor{bostonuniversityred}{rgb}{0.8, 0.0, 0.0}
\definecolor{brass}{rgb}{0.71, 0.65, 0.26}
\definecolor{lawngreen}{rgb}{0.49, 0.99, 0.0}
\definecolor{brightcerulean}{rgb}{0.11, 0.67, 0.84}
\definecolor{brightpink}{rgb}{1.0, 0.0, 0.5}
\definecolor{cadetblue}{rgb}{0.37, 0.62, 0.63}
\definecolor{cadmiumorange}{rgb}{0.93, 0.53, 0.18}
\definecolor{caputmortuum}{rgb}{0.35, 0.15, 0.13}
\definecolor{celadon}{rgb}{0.67, 0.88, 0.69}
\definecolor{chocolate(traditional)}{rgb}{0.48, 0.25, 0.0}
\definecolor{cinereous}{rgb}{0.6, 0.51, 0.48}
\definecolor{coolblack}{rgb}{0.0, 0.18, 0.39}
\definecolor{mediumspringbud}{rgb}{0.79, 0.86, 0.54}
\definecolor{lemonchiffon}{rgb}{1.0, 0.98, 0.8}
\definecolor{lightkhaki}{rgb}{0.94, 0.9, 0.55}
\definecolor{macaroniandcheese}{rgb}{1.0, 0.74, 0.53}
\definecolor{magicmint}{rgb}{0.67, 0.94, 0.82}

\definecolor{bittersweet}{rgb}{1.0, 0.44, 0.37}

\begin{document}
\begin{tikzpicture}
\begin{luacode*}
    -- setup
    
    -- hard-coded colours, which are therefore limited
    -- I am not sure these are sufficiently distinct or in a good order
    colours = {"lightgray","aureolin","amethyst","beaublue","applegreen","bleudefrance","atomictangerine","babypink","blizzardblue","bostonuniversityred","brass","lawngreen","brightcerulean","brightpink","cadetblue","cadmiumorange","caputmortuum","celadon","chocolate(traditional)","cinereous","coolblack","mediumspringbud","lemonchiffon","lightkhaki","macaroniandcheese","magicmint"}
    runoutofcolour = "bittersweet" -- backup colour used when colours is empy
    primecolours = {} -- used for collecting the prime colours in use

    -- ideally, target is divisible by rowlength, otherwise there is a partial row at the bottom
    -- suggested, either sqrt target (rounded down) to get rowlength, or fix rowlength and make target a multiple of it
    target = 100 -- how many to do
    rowlength=math.floor(math.sqrt(target)) -- used to arrange into a roughly-square grid
    -- rowlength = 10 -- fixed length row
    -- target = rowlength * 14
    
    -- picture is scaled as the number of digits of target increases so the inner circles are the right size for the numbers
    scale = 1.5 + string.len(target)/10 -- how spread apart the circles are
    scale2 = 1 + string.len(target)/10 -- how big the circles are
    
    -- quick sieve of the numbers up to target, storing their factors
    
    factorsof = {}
    
    factorsof[1] = {[1] = 1} -- 1 is weird
    primecolours[1] = table.remove(colours,1) -- lightgray
    
    for n=2,target do
        if factorsof[n] == nil then -- n is not current stored, so must be prime
            factorsof[n] = {[1] = n} -- store itself as its non-1 factor
            if colours[1] == nil then -- if run out of colours, use the backup colour
                primecolours[n] = runoutofcolour
            else -- use the next colour on the list
                primecolours[n] = table.remove(colours,1)
            end
            m=2 -- loop counter
            while m*n <= target do
                if factorsof[m] then -- loop over numbers already stored (which aren't necessarily sequential)
                    factorsof[m*n] = {} -- haven't got m*n stored because haven't got there yet
                    numfactors = #factorsof[m]
                    for i=1,numfactors do -- copy m's factors
                        factorsof[m*n][i] = factorsof[m][i]
                    end
                    factorsof[m*n][numfactors+1] = n -- add n as additional factor
                end
                m=m+1
            end
        end
    end
    
    -- draw the numbers
    
    for num,factors in ipairs(factorsof) do
        numfactors = #factors
        
        -- work out centre of this number and set as tikz \coordinate
        row = 1
        col = num
        while col>rowlength do
            row=row+1
            col=col-rowlength
        end
        tex.print("\\coordinate (" .. num .. ") at (" .. col*scale .. ",-" .. row*scale .. ");")
        
        if numfactors==1 then -- if prime, draw a full circle
            tex.print("\\draw[fill=" .. primecolours[num] .. "] (" .. num .. ") circle [radius=" .. 6*scale2 .. "mm];")
        else -- if not prime, draw segments
            tex.print("\\draw (" .. num .. ") circle [radius=" .. 6*scale2 .. "mm];") -- blank circle
            
            -- work out numfactors number of arcs
            arcsize = 360/numfactors
            theta=90 -- start at the top of the circle
            for arc = 1,numfactors do -- need an arc for each factor
                -- where to start the arc, bit of trig
                -- N.B. lua works in radians, but tikz works in degrees
                if theta <= 90 then
                    coords = "(" .. col*scale+0.6*scale2*math.cos(math.rad(theta)) .. ",-" .. row*scale-0.6*scale2*math.sin(math.rad(theta)) .. ")"
                elseif theta <= 180 then
                    coords = "(" .. col*scale-0.6*scale2*math.cos(math.rad(180-theta)) .. ",-" .. row*scale-0.6*scale2*math.sin(math.rad(180-theta)) .. ")"
                elseif theta <= 270 then
                    coords = "(" .. col*scale-0.6*scale2*math.cos(math.rad(theta-180)) .. ",-" .. row*scale+0.6*scale2*math.sin(math.rad(theta-180)) .. ")"
                else
                    coords = "(" .. col*scale+0.6*scale2*math.cos(math.rad(theta)) .. ",-" .. row*scale-0.6*scale2*math.sin(math.rad(theta)) .. ")"
                end
                
                -- now draw the arc
                tex.print("\\filldraw[fill=" .. primecolours[factors[arc]] .. "] (" .. col*scale .. ",-" .. row*scale .. ") -- " .. coords .. " arc (" .. theta .. ":" .. theta+arcsize .. ":" .. 6*scale2 .. "mm) -- (" .. col*scale .. ",-" .. row*scale .. ");")
                
                theta = theta+arcsize -- increase angle to next segment
            end
            
        end
        
        -- finally, draw a white circle in the middle with the number on it
        tex.print("\\draw[fill=white]  (" .. num .. ") circle [radius=" .. 3*scale2 .. "mm];")
        tex.print("\\node at (" .. num .. ") {" .. num .. "};")
        --tex.print("\\\\")
    end
\end{luacode*}
\end{tikzpicture}
\end{document}
